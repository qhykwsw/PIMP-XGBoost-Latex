% Template for producing ESWA-format journal articles using LaTeX
% Written by Miha Ravber
% Programming methodologies laboratory
% Faculty of Electrical Engineering and Computer Science
% University of Maribor
% Koroška cesta 46, 2000 Maribor
% E-mail: miha.ravber@um.si
% WWW: https://lpm.feri.um.si/en/members/ravber/
% Created: November 20, 2020 by Miha Ravber
% Modified: November 20, 2020 by Miha Ravber
% Use at your own risk :)
% Please submit your issues on the github page: https://github.com/Ravby/eswa-template


\documentclass[review]{elsarticle}
\graphicspath{ {./} }
\usepackage{hyperref}
\usepackage{float}
\usepackage{verbatim} %comments
\usepackage{apalike}
\usepackage{mathtools}
\usepackage[ruled,vlined]{algorithm2e}
\usepackage{booktabs}
\usepackage{multirow}
\usepackage[normalem]{ulem}
\useunder{\uline}{\ul}{}
\usepackage{amssymb}
\usepackage{array}
\usepackage{geometry}
    \geometry{
    a4paper,
    total={170mm,257mm},
    left=20mm,
    top=20mm,
}

\restylefloat{figure}
\restylefloat{table}

\journal{Expert Systems with Applications}

% For ESWA journal you need to use APA style
\bibliographystyle{model5-names}\biboptions{authoryear}


\begin{document}


\section{Supplementary materials}
\subsection{Hyperparameters of Models}
\subsubsection{Logistic regression}

\begin{table}[H]
    \begin{tabular}{|m{9em}|m{14em}|}
    \hline
    \textbf{Parameters} & \textbf{Values} \\ \hline
    solver & newton-cg \\ \hline
    penalty & none \\ \hline
    the others & default in scikit-learn package \\ \hline
    \end{tabular}
\end{table}

\subsubsection{Artificial neural network}

\begin{table}[H]
    \begin{tabular}{|m{9em}|m{14em}|}
    \hline
    \textbf{Parameters} & \textbf{Values} \\ \hline
    hidden\_layer\_sizes & (64, 64) \\ \hline
    batch\_size & 64 \\ \hline
    learning\_rate\_init & 0.001 \\ \hline
    the others & default in scikit-learn package \\ \hline
    \end{tabular}
\end{table}

\subsubsection{Support vector machine}

\begin{table}[H]
    \begin{tabular}{|m{9em}|m{14em}|}
    \hline
    \textbf{Parameters} & \textbf{Values} \\ \hline
    kernel & poly \\ \hline
    degree & 4 \\ \hline
    the others & default in scikit-learn package \\ \hline
    \end{tabular}
\end{table}

\subsubsection{Decision tree}

\begin{table}[H]
    \begin{tabular}{|m{9em}|m{14em}|}
    \hline
    \textbf{Parameters} & \textbf{Values} \\ \hline
    criterion & gini \\ \hline
    max\_depth & 7 \\ \hline
    min\_samples\_split & 5 \\ \hline
    min\_samples\_leaf & 0.05 \\ \hline
    the others & default in scikit-learn package \\ \hline
    \end{tabular}
\end{table}

\subsubsection{Bagging}

\begin{table}[H]
    \begin{tabular}{|m{9em}|m{14em}|}
    \hline
    \textbf{Parameters} & \textbf{Values} \\ \hline
    max\_features & 0.5 \\ \hline
    max\_samples & 1 \\ \hline
    n\_estimators & 660 \\ \hline
    the others & default in scikit-learn package \\ \hline
    \end{tabular}
\end{table}

\subsubsection{AdaBoost}

\begin{table}[H]
    \begin{tabular}{|m{9em}|m{14em}|}
    \hline
    \textbf{Parameters} & \textbf{Values} \\ \hline
    n\_estimators & 200 \\ \hline
    learning\_rate & 0.1 \\ \hline
    the others & default in scikit-learn package \\ \hline
    \end{tabular}
\end{table}

\subsubsection{Random forest}

\begin{table}[H]
    \begin{tabular}{|m{9em}|m{14em}|}
    \hline
    \textbf{Parameters} & \textbf{Values} \\ \hline
    n\_estimators & 180 \\ \hline
    max\_depth & 16 \\ \hline
    min\_samples\_split & 5 \\ \hline
    min\_samples\_leaf & 1 \\ \hline
    max\_features & 3 \\ \hline
    the others & default in scikit-learn package \\ \hline
    \end{tabular}
\end{table}

\subsubsection{Gradient boosting decision tree}

\begin{table}[H]
    \begin{tabular}{|m{9em}|m{14em}|}
    \hline
    \textbf{Parameters} & \textbf{Values} \\ \hline
    n\_estimators & 1500 \\ \hline
    learning\_rate & 0.01 \\ \hline
    max\_depth & 11 \\ \hline
    min\_samples\_split & 10 \\ \hline
    min\_samples\_leaf & 2 \\ \hline
    max\_features & 8 \\ \hline
    the others & default in scikit-learn package \\ \hline
    \end{tabular}
\end{table}

\subsubsection{XGBoost}

\begin{table}[H]
    \begin{tabular}{|m{9em}|m{14em}|}
    \hline
    \textbf{Parameters} & \textbf{Values} \\ \hline
    n\_estimators & 900 \\ \hline
    learning\_rate & 0.01 \\ \hline
    max\_depth & 9 \\ \hline
    colsample\_bytree & 0.5 \\ \hline
    subsample & 0.8 \\ \hline
    gamma & 0.2 \\ \hline
    reg\_alpha & 0.01 \\ \hline
    reg\_lambda & 1 \\ \hline
    the others & default in XGBoost package \\ \hline
    \end{tabular}
\end{table}

\subsubsection{LightGBM}

\begin{table}[H]
    \begin{tabular}{|m{9em}|m{14em}|}
    \hline
    \textbf{Parameters} & \textbf{Values} \\ \hline
    n\_estimators & 700 \\ \hline
    learning\_rate & 0.01 \\ \hline
    max\_depth & 7 \\ \hline
    num\_leaves & 50 \\ \hline
    max\_bin & 100 \\ \hline
    min\_child\_samples & 30 \\ \hline
    min\_child\_weight & 0.001 \\ \hline
    colsample\_bytree & 0.8 \\ \hline
    subsample & 0.8 \\ \hline
    reg\_alpha & 1 \\ \hline
    reg\_lambda & 0.7 \\ \hline
    the others & default in LightGBM package \\ \hline
    \end{tabular}
\end{table}

\end{document}
